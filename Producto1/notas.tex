% Ejemplo de documento LaTeX
% Tipo de documento y tamaño de letra
\documentclass[12pt]{article}

% Preparando para documento en Español.
% Para documento en Inglés no hay que hacer esto.
\usepackage[spanish]{babel}
\selectlanguage{spanish}
\usepackage{setspace}
\usepackage[utf8]{inputenc}

% EL titulo, autor y fecha del documento
\title{Tutorial breve de los comandos de Bash}
\author{Ana Gabriela Carretas Talamante}
\date{05 de Febrero de 2015}

% Aqui comienza el cuerpo del documento
\begin{document}
% Construye el título
\maketitle

\section{¿Qué es {\tt bash}?}

Bash es un interpretador de comandos utilizado sobre el sistema operativo Linux.
Su función es de mediar entre el usuario y el sistema.

\section{Las líneas de Comando}

Algo que hace especial a Linux es su interfaz gráfica, esto hace que el usuario se sienta un poco más familiarizado con otros sistemas operativos.Linux utiliza una terminal, está conformada por líneas de comando y en ella se teclean las operaciones que ha de realizar la computadora. Una terminal se abre con {\tt Ctrl+Alt+T} 

\section{Comandos básicos}


\subsection{Para navegar}
\begin{tabular}{|p{3cm}|p{10cm}|p{2cm}|}
\hline
Comando & Descripción & Ejemplo \\
\hline
{\tt echo \$SHELL} & Muestra una variable del sistema que incluye a Shell. & echo \$SHELL \\ \hline
{\tt pwd} & Muestra el directorio donde estamos trabajando. & pwd  \\ \hline
{\tt ls} & Enlista los contenidos del directorio. & ls Documents \\ \hline
{\tt cd} & Change Directories - ie. Moverse a otro directorio. & cd AAA\\ \hline
\end{tabular} 

\subsection{Acerca de los archivos y directorios}

\begin{tabular}{|p{3cm}|p{10cm}|p{2cm}|}
\hline
Comando & Descripción & Ejemplo \\
\hline
{\tt ls -a.} & Enlista los contenidos de un directorio, incluyendo a los archivos ocultos. & ls -a \\ \hline
{\tt file} & Muestra qué tipo de archivo es un archivo o un directorio. & file \\ \hline
\end{tabular} 

\subsection{Una para manuales}

\begin{tabular}{|p{3cm}|p{10cm}|p{2cm}|}
\hline
Comando & Descripción & Ejemplo \\
\hline
{\tt man -comando-} & Muestra una lista de comandos, y dentro de ella encuentra el que se tecleó. & man ls \\ \hline
{\tt man -k -término-} & Busca en todos los manuales de comandos lo que contenga cada símbolo escrito. & man -k ls \\ \hline
\end{tabular} 

\subsection{Manipulación de archivos}
\begin{tabular}{|p{3cm}|p{10cm}|p{2cm}|}
\hline
Comando & Descripción & Ejemplo \\
\hline
{\tt mkdir} & Crear un directorio. & mkdir bbb \\ \hline
{\tt rmdir} & Borrar un directorio. & rmdir bbb \\ \hline
{\tt touch} & Crea un archivo en blanco. & touch white \\ \hline
{\tt cp} & Copia un archivo o un directorio. & cp aaa white \\ \hline
{\tt mv} & Mueve un archivo o directorio, (también se puede usar para cambiar el nombre). & mv aaa white \\ \hline
{\tt rm} & Elimina un archivo. & rm y \\ \hline
\end{tabular} 

\subsection{Vi: el editor de textos en la terminal}
\begin{tabular}{|p{3cm}|p{10cm}|p{2cm}|}
\hline
Comando & Descripción & Ejemplo \\
\hline
{\tt vi} & Edita un archivo. & vi y \\ \hline
{\tt cat} & Muestra un archivo. & cat y \\ \hline
{\tt less} & Compacta la vista para ver archivos largos. & less y \\ \hline
\end{tabular} 

\subsection{Permisos en general}
\begin{tabular}{|p{3cm}|p{10cm}|p{2cm}|}
\hline
Comando & Descripción & Ejemplo \\
\hline
{\tt chmod} & Cambia los permisos de un archivo o directorio. & chmod -w y \\ \hline
{\tt ls -ld} & Muestra los permisos de un directorio respectivo. & ls -ld y \\ \hline
\end{tabular} 

% Nunca debe faltar esta última linea.
\end{document}